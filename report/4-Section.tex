\newpage
\section{Factory level}
\subsection{Factory level energy consumption}
\label{chapter 4}

\textit{Person in Charge; Md Sahidul Islam and Wzheng}
%---- the scope of this chapter}}

\subsubsection{Scope}
Energy efficiency has developed into an important objective for industrial enterprises. However, there is still a need for systematic approaches to reduce energy consumption in factories. From an economic point of view, industrial enterprises have an incentive to reduce their energy consumption because of increasing energy prices, such as the European average prices for gas in industry, which rose by approximately 34\% during the last four years \cite{European2013}. 
Despite this situation, the implementation of energy efficiency measures has not met the expectations yet. The reasons for the deficits in realizing energy efficiency include lack of time, lacking transparency on energy consumption, lacking capital for investments and divided responsibilities within a company \cite{Fleiter2013}. 

Product design
Process design
Process adjustments:
Post-processing
plan and implementations


\subsubsection{define factory level}
Different tools and methods have been developed in recent years to support the systematic analysis and optimization of industrial enterprises for reducing their energy consumption. However, the existing methods mainly focus on product design, Process design, process adjustment, manufacturing processes, machine process and smart factory building design. Although these are important aspects of the energy-efficient factory, considering the interrelationships between products, processes and resources in the factory system is essential for a holistic integration of energy efficiency in the enterprise. 

\subsubsection{factory building energy consumption }
The starting point for the approach is the definition of the project task or planning situation by the factory planning participant (user input). The most important parameters to describe the task are object level, system process, part of the energy chain, energy form, planning case and user’s role. According to their background, the first four parameters are defined as technical parameters and the last two as organizational parameters. 

The object level describes the level of abstraction of the considered system (e.g. factory, building, plant area, single machine). The system process defines the process of the enterprise to which the considered system belongs to (e.g. assembly, logistics). The part of the energy chain describes whether the system performs energy generation, conversion, distribution, storage or use, since factories increasingly integrate several of these functions \cite{Muller2013a}. The energy form defines the types of resources that are used within the considered system (e.g. electricity, water). The planning case comprises the extent to which changes are possible in the system; planning a new system has the highest degrees of freedom, whereas operating the existing system equals the lowest degree of freedom. Finally, the user's role defines the perspective of the user (e.g. factory planner, worker). 

the implementing methods is the high effort for data acquisition. Therefore, an approach to reduce energy consumption within factory systems was developed that provides energy efficiency measures to factory planning participants based on qualitative data \cite{Muller2013}.

Herrmann et al.\cite{Herrmann2011} proposed a holistic definition of factory management, including technical building services as an important player \cite{Herrmann2011}. Seow and Rahimifard outlined a framework for modelling energy consumption within a manufacturing system from a product’s viewpoint, and utilized the energy data at factory and process levels \cite{Seow2011}. Mouzon et al. proposed a mathematics programming model to optimize the total energy consumption of manufacturing \cite{Gilles2007}. Clearly, a comprehensive energy consumption models at this level is still lacking. 

Manuela Krones and Egon Müller \cite{Krones2014} developed a general concept has been developed to systematically guide a factory planning participant from his or her project task to appropriate energy efficiency measures. We introduce their work in detail here. The goal is to provide suitable energy efficiency approaches in order to increase the efficiency of information gathering. The approach consists of four major steps, which are explained in the following.


When applying the method, not all of these parameters need to be specified. The user can choose which parameters to specify; however, if the number of specified parameters is too small, the user may receive too unspecific results and needs to repeat the approach with changes in the input. 

\begin{figure}[h!]
	\centering
	\includegraphics[width=0.6\linewidth]{Figure/Overall-concept-methods.jpg}
	\caption{Overall concept for methodical approach to systematically identify energy efficiency measures}
	\label{fig:overalconcept}
\end{figure}


\subsection{Smart Factory Building}
Firstly did research on the relation between green buildings, smart buildings, and energy efficiency. To realize how they relate each other together with energy usage, and consequently putting deeper attention on the factors that have a direct impact on consumption. We also generated an economic model based on information concerning: energy consumption and distribution in different types of buildings, prices of energy, average savings due to the implementation of smart technologies, and prices of smart technologies. 
Here we will mention some interesting energy efficiency measures that are strictly related to the energy consumption in buildings and that would be close to composing all the aspects related with that in a project like this. As figure shows a general classification of the influencing systems.

\begin{figure}[h!]
	\centering
	\includegraphics[width=0.8\linewidth]{Figure/FacturyEnergyConcumption.png}
	\caption{Smart factory building design conceptual approach}
	\label{fig:factoryenergyconsumption}
\end{figure}


\subsection{Heating, ventilation, and air conditioning (HVAC)}

\subsubsection{Heating }
Heaters are appliances whose purpose is to generate heat (i.e. warmth) for the building. This can be done via central heating. Such a system contains a boiler, furnace, or heat pump to heat water, steam, or air in a central location such as a furnace room in a home, or a mechanical room in a large building. The heat can be transferred by convection, conduction, or radiation. Since the new AHU system has heat recovery, this section takes into account that 80% of the supply air is already heated exhaust air. 

\subsubsection{Ventilation }
Ventilation is the process of changing or replacing air in any space to control temperature or remove any combination of moisture, odors, smoke, heat, dust, airborne bacteria, or carbon dioxide, and to replenish oxygen. Ventilation includes both the exchange of air with the outside as well as circulation of air within the building. It is one of the most important factors for maintaining acceptable indoor air quality in buildings. Methods for ventilating a building may be divided into mechanical/forced and natural types.The energy consumption in a single air handling unit, or process, can be noticed in the
enthalpy’s change over time.

\subsubsection{Air Conditioning }
An air conditioning system, or a standalone air conditioner, provides cooling and humidity control for all or part of a building. Air conditioned buildings often have sealed windows, because open windows would work against the system intended to maintain constant indoor air conditions. Outside, fresh air is generally drawn into the system by a vent into the indoor heat exchanger section, creating positive air pressure. The percentage of return air made up of fresh air can usually be manipulated by adjusting the opening of this vent. Typical fresh air intake is about 10%.

\subsubsection{Cooling}
cooling systems can have very high efficiencies, and are sometimes combined with seasonal thermal energy storage so that the cold of winter can be used for summer air conditioning. Common storage mediums are deep aquifers or a natural underground rock mass accessed via a cluster of small-diameter, heat-exchanger-equipped boreholes. Some systems with small storage's are hybrids, using free cooling early in the cooling season, and later employing a heat pump to chill the circulation coming from the storage. The heat pump is added-in because the storage acts as a heat sink when the system is in cooling (as opposed to charging) mode, causing the temperature to gradually increase during the cooling season.

recovery: The use of heat/enthalpy wheels and energy recovery ventilator allow them to absorb moisture from the air while at the same time cooling the air that is absorbed, to finally exhaust heated air. It is a system that allows the capacity of the HVAC system to be reduced since it can be used in both summer andwinter months. In summer it would take the heat and humidity outside the building while in winter it would exhaust the recovered heat inside the building (2015). Infrared heaters: These alternative devices can be powered electrically, by propane or with natural gas. The increase in efficiency is due to the higher impassivity the can produce compared with traditional heaters, even if in most of the cases their combustion efficiency is lower. Solar systems: Using the sun to heat water and use this thermal storage to heat the building is both cost and energy efficient. Cooling systems: When talking about this, we have to also mention the refrigeration systems. And there are three things when considering efficiency in these systems: energy usage, type and quantity of heat exhausted and refrigerant type. The air-cooled systems are the most spread used and as in many cases, the traditional equipment is not optimal in terms of energy efficiency (2017).

\begin{table}[H]
\centering
\caption{Definition of symbols \cite{Watson2018}}
\label{table:symbols}
\begin{tabular}{llll}
\hline
& 2020 & 2030 & 2050 \\ \hline
Greenhouse gas reduction & 20\% & 40\%  & 80\% \\
Energy production from renewable & 20\% & 27\% & \\
Energy efficiency improvement & 20\% & 27\% & \\ \hline
\end{tabular}

\end{table}

Energy efficiency improvement Emissions reductions in the last years have been reduced mainly thanks to progress on renewable and on energy efficiency measures. However, from the three targets of the European Union, the only one that will not be achieved by 2020 will be the one regarding improvement in energy efficiency (European Commission, 2017 \cite{European2013}). It is due to a number of barriers that go out of the scope of this project, but in general, they are related with financial uncertainty, cultural behaviors and lack of knowledge on energy efficiency measures.

We need to investigate towards several directions in order to make factories more energy efficient. General issues to be solved include:

\begin{itemize}  
\item An inter operable infrastructure that allows us to on-line monitor energy consumption down to discrete device level
\item Concrete models for energy consumption prediction at each layer e.g. device level, location, process level etc
\item Enterprise services that evaluate and assist in optimizing plans and processes dynamically based on their energy usage e.g. at production, at supply chain, etc
\item Applicability of Market driven mechanisms and close collaboration with energy providers for optimal energy usage / Integration of the distributed alternative energy resources
\item Support for software management for large scale infrastructures e.g. remote monitoring, remote diagnostics,
\end{itemize}

The application when a technician is accessing the main view of the building. Here the five different administrative domains can be accessed by pressing on the desired floor. Moreover, important events are listed on the left part of the window and the user can click on them to directly access the device emitting the alert.

\subsection{<>Conclusion }
When exceeding the level of milt machine chains, simulation techniques become predominant to master the complexity of predicting energy and resource flows for larger manufacturing systems.
Technical building services can consume considerable amounts of energy:
It is obvious that factory layout and facility optimization need careful attention in a design stage.
Just like residential buildings, production facilities need to be constructed according to state of the art building physics principles, thus minimizing energy inputs for HVAC   conditioning of the work environment, while taking into account local climate conditions.
Similar to the multi-machine strategy, production planning can be optimized at a facility wide level in order to limit the total energy consumption.

