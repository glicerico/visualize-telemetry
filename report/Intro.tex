


%how to cite
%\cite{Seow2011}

%how to add figure
% \begin{figure}[h!]
% 	\centering
% 	\includegraphics[width=0.8\linewidth]{Figure/Total-consumption.jpg}
% 	\caption{Total Consumption by End-Use Sector, 1949-2011 \cite{Apostolos2013}}
% 	\label{fig:TotalConsumption}
% \end{figure}



\newpage
\section{Introduction}

The current project explores a method to predict anomalous events in the dynamics of a network, via finding their causes.
The main idea in the method is to find the grammar of the network -- its implicit dynamical structure-- in an unsupervised manner, starting from telemetry data of its different components.
If such a grammar can be obtained, then it would be possible to predict (with a certain confidence) an anomaly when observing dynamics that preceed such a state in the grammar.
This would be similar to a situation in language processing where, after observing a sequence of words that form a sentence's subject (e.g. determiner-adjective-noun), we can expect a verb to follow.\\

This project will leverage SingularityNet's Unsupervised Language Learning (ULL) pipeline, which attempts to find the grammar implicit in a given corpus of sentences.
In order to process the network data with such pipeline, the process can be divided into three stages:
\begin{enumerate}
\item Abstract the network dynamics by converting the state of the network at each timestep into a real-valued vector (or a set of vectors).
\item Using symbolic dynamics techniques, convert the dynamics of the network embedding vector(s) into a sequence of symbols; these sequences would be functionally equivalent to a natural language corpus in the ULL project.
\item Apply a suitable version of the ULL pipeline to the sequences obtained in step 2 (the sentences), in order to learn the network grammar.
\end{enumerate}

The first step could be done in a number of ways.
One popular and successful way to create distributed representations is using Neural Networks.
Deep Neural Networks like graph2seq have been used to abstract graphs, with some success 
\cite{venkatakrishnan_graph2seq_2018}.\\
TESFA, YOU CAN ADD SOME STUFF HERE.

Another approach for creating embeddings is to use the features as they come from the telemetry data.
We follow a similar processing as done by \cite{putina_telemetry-based_2018}.

For step 3, we need to adapt the ULL pipeline to our purposes: instead of having a natural language sentence where every word is presented to the algorithm at the same time, we would be dealing with a continous stream of tokens representing the state of the network at different times.
In particular, the parser used in the ULL pipeline needs to be replaced by one that handles a continous stream of tokens.
For this purpose, we implemented a continuous version of the MST-parser proposed by Yuret \cite{Yuret1999}\footnote{Code available at https://github.com/glicerico/stream-parser}
