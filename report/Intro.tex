


%how to cite
%\cite{Seow2011}

%how to add figure
% \begin{figure}[h!]
% 	\centering
% 	\includegraphics[width=0.8\linewidth]{Figure/Total-consumption.jpg}
% 	\caption{Total Consumption by End-Use Sector, 1949-2011 \cite{Apostolos2013}}
% 	\label{fig:TotalConsumption}
% \end{figure}



\newpage
\section{Introduction}

The current project explores a method to predict anomalous events in the dynamics of a network, via finding their causes.
The main idea in the method is to find the grammar of the network -- its implicit dynamical structure-- in an unsupervised manner, starting from telemetry data of its different components.
If such a grammar can be obtained, then it would be possible to predict (with a certain confidence) an anomaly when observing dynamics that preceed such a state in the grammar.
This would be similar to a situation in language processing where, after observing a sequence of words that form a sentence's subject (e.g. determiner-adjective-noun), we can expect a verb to follow.\\

This project will leverage SingularityNet's Unsupervised Language Learning (ULL) pipeline, which attempts to find the grammar implicit in a given corpus of sentences.
In order to process the network data with such pipeline, the process can be divided into three stages:
\begin{enumerate}
\item Abstract the network dynamics by converting the state of the network at each timestep into a real-valued vector (or a set of vectors).
\item Using symbolic dynamics techniques, convert the dynamics of the network embedding vector(s) into a sequence of symbols; these sequences would be functionally equivalent to a natural language corpus in the ULL project.
\item Apply a suitable version of the ULL pipeline to the sequences obtained in step 2 (the sentences), in order to learn the network grammar.
\end{enumerate}

As noticed by @ben, the parser used for this should be able to handle a continuous stream of tokens.

Now it is obvious that the industrial industry consumed most of the energy in earth since 1950s, so study the energy consumption in is very important and meaningful. Today, energy efficiency in production systems has partially been achieved on the component level, also several methods for the energy optimal operation of plants, machines and components.  Some articles propose a novel generic method to model the energy consumption behavior of machines and plants based on a statistical discrete event formulation. Manufacturing is the major processes of manufacturing industry and it uses one or more physical mechanisms, Figure \ref{fig:Manufacturingprocess} to transform material form or shape which requires energy inputs.  Manufacturing companies commune a lot of energy in their daily productions, including fossil fuels, electricity etc.

\begin{figure}[h!]
	\centering
	\includegraphics[width=0.6\linewidth]{Figure/Manufacturing-process.jpg}
	\caption{Manufacturing process from an energy point of view \cite{Apostolos2013}}
	\label{fig:Manufacturingprocess}
\end{figure}

The energy usage cost money and cause environmental concerns \cite{Dietmair2009}. Also the development of environmentally friendly products gains an increasing importance in science and in industry \cite{Abele2005}. Recently, many countries have proposed plans for energy consumption reduction. The environmental awareness leads the EU member states agreeing on the principle of ''20/20/20 by 2020'', including a 20\% reduction in greenhouse gases, a 20\% share of renewable energies and a 20\% increase in energy efficiency by the year 2020 as compared to 1990 indicators \cite{Apostolos2013}. The major country in Europe, Germany also adopted the Energy Concept 2050 plan, while China brought up new methods for energy saving and emissions reduction in 12th five year plan. The first unit of the chapter proposed the overall situation about the energy consumed in industry. Next chapters will mention the overall study for energy consumption simulation and way of reducing energy cost, and analyze it from factory to machine level.


\subsection{The energy consumption on different levels}
\subsubsection{Machine level}

The first part of the chapter focus on the energy consumption on more macroscopic aspects. However to summarize the latest development of research for energy consumption. The consumed energy of a machine is split up in to 3 parts, active energy, additional energy requirement of the machine, and the energy demand of the process-periphery.  For example Duflou et al.\cite{Duflou2012} provide a systematic overview of the state of the art in energy and resource efficiency increasing methods and techniques, in the domain of discrete part manufacturing, with attention being given to the effectiveness of the available options.

Optimization strategies is a very important work for reducing machine energy consumption, including find more efficient machine tool components and Waste recovery. The modelling framework for tool machine energy consumption forecasting is studied. A number of examples have been given on how this can be applied for energy efficiency optimization. Energy consumption forecasting and optimization for tool machines,   the energy saving potentials in each method can be estimated according to real body-shop production characteristics. Optimizing the production scheduling of a single machine is able to minimize total energy consumption costs. Energy efficient production planning, sustainable and green manufacturing. A relatively high saving potential can be based on the reuse of recuperated kinetic energy, but there is still challenges to find cost-effective and safe solutions. The generic energy consumption model for machine is also suitable to provide the basis for the energy optimal use of manufacturing plants and factories.




\subsubsection{The energy consumption on manufacturing line }

Manufacturing line contains assembly line and production line, it is a set of important sequential operations established in a factory. Many paper develops operational methods for the minimization of the energy consumption of manufacturing equipment. Classical studies \cite{Gungor1999} mentioned the goal of many modern manufacturers is to decrease the cost of energy production by any means while satisfying the environmental regulations and ensuring quality, and customer satisfaction. In order to facilitate the process of replacing existing energy sources with sustainable alternatives and to minimize the environmental impact, the energy efficiency in manufacturing has to be improved, modeling framework for production machine energy consumption forecasting has been presented in Anton Dietmair’s group, and the utilization of energy flows. Research on optimization and merging with numerical control simulation to maximize the energy efficiency of industrial machine production. 


\subsubsection{The energy consumption on factory level}

This part studies the scope, energy simulation model process and the way of management. Forecasting energy consumption of multi-family residential buildings using support vector regression, include investigating the impact of temporal and spatial monitoring granularity on performance accuracy. The energy optimization potential of the overall process or the total system are exploited through a holistic view of the complex interactions of individual resources, processes and structures of a factory. Including forming an energy simulation model for factory between process planning and manufacture. Such as approaching for improve based energy management in smart factory, it is obvious that factory layout and facility optimization need careful attention in a design stage. It is obvious that factory layout and facility optimization need careful attention in a design stage and the final goal is to define methods for improving energy consumption on factory level. Due to the importance of calculating and reducing energy consumption for residential buildings. Work also directed towards extending this methodology, optimization and merging with numerical control simulation to provide strong tools for the digital factory.

\subsection{Summary}

Most energy consumption studies focus on two directions, the macroscopic way contains the amount of total energy estimation cost of the whole industry. The huge energy consumption in manufacturing industry is studied and analyzed. The research also focused on the major production activities in factories include forming, machining processes, primary shaping processes, assembly processes, and the summary of energy cost in building and factory level. No matter what process it is mentioned, the goal is model and simulate the energy consumption of machines/factories accurately and reduce the energy cost for saving money and environment protection. The remainder of the paper is structured as follows: Section. 1 introduction to energy consumption. In Section. 2, the Machine level for energy consumption are sketched. Section 3 presents the energy consumption study of Multi-machine level. Section. 4 focus on the more macroscopic Factory level energy consumption reducing study. Finally, conclusions about the review are given in Section.5.

\begin{figure}[h!]
	\centering
	\includegraphics[width=0.9\linewidth]{Figure/Framework-review.jpg}
	\caption{Framework of review}
	\label{fig:FrameworkReview}
\end{figure}
