\newpage
\section{Machine level}
\label{chapter2}
\textit{}
%---- the scope of this chapter}}

The discrete part manufacturing processes will be discussed in detail. It is defined as production processes in which the output can be identified and is measurable in distinct units\cite{Duflou2012}. Manufacturing processes are already widely described in previous literature. Unit process boundaries are characterized through Figure \ref{fig:BoundariesProcess}. A unit process can be described as an individual machine tool.


%---- Define machine level 

A sequence of unit processes make up a general production process, a unit process should be clearly defined where it starts and where the process ends, or in other words define all the boundaries to the environment and other unit processes. Machine level is not the lowest level, also sub-unit level studies need to be performed\cite{Kellens2012}.

\begin{figure}[h!]
	\centering
	\includegraphics[width=0.8\linewidth]{Figure/Unit-process.JPG}
	\caption{System boundaries of a unit process \cite{Kellens2012}}
	\label{fig:BoundariesProcess}
\end{figure}

In chapter \ref{chapter2} energy consumption of machine tools during there use stage is discussed. Various processes are analyzed and collected in a life cycle inventory (LCI) databases. Different industries or entities make a LCI. The databases from LCI can give good descriptions for material production and manufacturing processes but sometimes lack in overall environmental impact\cite{Duflou2012}. Examples of LCI databases are; Ecoinvent database \cite{Frischknecht2007} European Aluminium Association \cite{EuropeanAluminiumAssociation2018} and Plastics Europe \cite{EcoProfiles}. There are a lot of different manufacturing processes investigated by the LCI databases. These databases can cover a full process taxonomy like DIN8580 \cite{DIN2003}. But LCI databases have several shortcomings that improve every year, e.g. it being limited to primary material production or only limit to more conventional processes. 

%-------    CO2PE!-initiative

Performing life cycle analysis (LCA) is necessary for the further growth of energy efficient manufacturing. This growth is written down in an initiative to better coordinate all the research. The CO2PE!-Initiative \cite{Kellens2012} has as the objective to coordinate international efforts aiming to document and analyse the environmental impacts of a wide range of current and emerging manufacturing processes, and to provide guidelines to reduce these impacts.

The databases provide information about manufacturing processes which can be used to calculate a complete production line to discuss total energy consumption and so on. Only basic energy energy parameters are provided by these databases, the concept is explained in the General machine section. Besides a basic model for energy consumption specific processes  also need to examine, these include forming processes, substractive processes, additive processes and industrial robots.

\subsection{General machine}

The energy calculation for life cycle inventory is basically similar for all the production processes. The consumed energy of a machine can be split up in to three parts according to Abele et al.\cite{Abele2005}.
\begin{itemize}  
\item $E_{th}$, the active energy 
\item $E_{additional}$, the additional energy requirement of the machine 
\item $E_{periphery}$, the energy demand of the process-periphery 
\end{itemize}

The active energy $E_{th}$ incorporates the theoretical energy needed to perform the change of shape. It represents the minimum energy consumption of a machine. 

The transformation energy is only a fraction of the total energy used by the machine. Additional energy is calculated with equation \ref{eq:Eadditional}. It includes the basic power $P_{basic}$ and the average idle power $P_{idle}$ with respectively the basic time $t_{b}$and the work-piece utilization time $t_{U}$.

\begin{equation} \label{eq:Eadditional}
E_{additional} = P_{idle} \cdot t_b + P_{basic} \cdot (t_{U} - t_{b})
\end{equation}

Also the process periphery is used in the energy calculations. It is an summation of all the single electricity consumers multiplied by the work piece utilization time.

\begin{equation} \label{eq:Eperiphery}
E_{periphery,el} = \sum_{i} P_{i} \cdot t_U
\end{equation}

In most studies a general concept based on machines states is used, see Figure \ref{fig:energymachine}. Schrouf et al.\cite{Shrouf2014} and Bajpai et al.\cite{Bajpai2018} are two examples who use this concept. These studies all perform energy consumption calculations on production line level, which is explained in chapter \ref{chapter3}. Each state has it's own time value depending on the production line and a power value.

\begin{figure}[h!]
	\centering
	\includegraphics[width=0.6\linewidth]{Figure/energymachine.JPG}
	\caption{A schema for machine states and transition \cite{Shrouf2014}}
	\label{fig:energymachine}
\end{figure}

%---- How to detirmine the LCI


\subsection{Energy reduction strategies}

Optimizing a single machine is done by the machine tool manufacturers. They optimize the machine tool design, this is different from process planners who optimize the entire process through the process parameters\cite{Duflou2012}. Energy reduction begins with a good selection of the needed tools or also considering substituting a tool when necessary.\newline

\textbf{Optimized machine tool design}

\begin{itemize}  
\item \textit{More efficient machine tool components}, improving a machine can obliviously be done by improving the components efficiency. Examples are more efficient drives, pumps, spindles, etc.
\item \textit{Technological}, instead of just improving the components efficiency. Adopting a new technology could lead to an significant gain. For example replacing the conventional C02-lasers with a new generation of diode lasers.
\item \textit{Waste recovery within a machine tool}, energy improvement can also be achieved by recovering energy to reuse. Diaz et al. \cite{Diaz2009} uses energy recovery in a milling machine through kinetic energy recovery system (KERS). They improved the energy consumption up to 25\%.
\end{itemize}


\textbf{Effects of optimized process control}
\begin{itemize}  
\item \textit{Selective actuation of non-continuously required devices}, selectively shutting down devices of which the functionality is not required in specific operations. Example: switching to a less energy consuming stand-by level.
\item \textit{Reducing idle production times}, the second parameter in energy demand is time. Reducing the unproductive idle time during a manufacturing can lead to an overall increase energy efficiency.
\item \textit{Optimized process parameters}, a significant decrease in energy consumption can be obtained through the right selection of process parameters. Examples is increasing cutting speed when milling, thereby shortening the machining time and reducing the energy. 
\item \textit{Energy and resource efficient process modeling and planning}, Various models for energy, resource efficient process modeling, planning and scheduling have been presented in literature. 
\end{itemize}

\textbf{Effects of process/machine tool selection}

Selecting a certain tool is always done by an comparison. Energy is not the only selection parameter for comparing machines but is certainly very important. Different studies have already been performed on comparing different types of machines. Yoon H. et al. \cite{Yoon2014} did a comparison of energy consumption between bulk forming, subtractive, and additive processes. Others like Watson J. et al. \cite{Watson2018} made a decision-support model for selecting additive manufacturing versus subtractive manufacturing based on energy consumption.

\subsection{Forming Processes}

Forming processes are manufacturing processes which make use of suitable stresses (like compression, tension, shear or combined stresses) which cause plastic deformation of the materials to produce required shapes. 

\begin{figure}[H]
	\centering
	\includegraphics[width=0.8\linewidth]{Figure/formingprocess.jpg}
	\caption{overall forming process}
	\label{fig:formingprocess}
\end{figure}

The current technology to produce sheet metal parts in high quantities is primarily based on the use of mechanical forming presses. A common design is the eccentric drive in addition to a flywheel which is driven by an asynchronous electric motor. The flywheel is coupled with the eccentric shaft via a clutch and the ram is mechanically connected to the eccentric shaft by means of conrods. The eccentric bearing of the conrods leads to a sinusoidal motion of the ram as shown in Figure \ref{fig:formingdesign} \cite{Behrens2016}.

\begin{figure}[H]
	\centering
	\includegraphics[width=0.8\linewidth]{Figure/ramdesign.jpg}
	\caption{Common design of forming part of the process}
	\label{fig:formingdesign}
\end{figure}

The ram's stroke rate is regulated by the rotational motor speed with a frequency converter. In order to realize a high output rate automated press lines are used, in which the sheet metal is fed from a coil by means of sheet metal feeders.

\subsubsection{Possible improvements}

In the forming processes the forming force is very high. For instance, in the automobile panel forming process the forming force is nearly up to 9000kN, but the machine energy efficiency of the drawing process is low to 7\%; the main reason is the mismatch between the output power of the equipment and the demanded power of the process, which leads to large energy consumption. Therefore, there is a desire for models or tools for energy consumption reduction and energy efficiency improvement \cite{Gao2018}.

\textbf{Metal feeding}

New energy-efficient drive concepts for forming presses and sheet metal feeding systems are developed. The novel press drive is based on a power-split design, which allows a variable ram-kinematics with reduced total costs of ownership in comparison to conventional servo presses. The new feeding concept will be able to realize the contactless feed of electrically conductive sheet metals by means of electromagnetic forces. Since only the sheet metal must be accelerated, the energy efficiency and feeding rate can be increased significantly \cite{Behrens2016}.

\textbf{Stamping}

Stamping is an important sheet metal forming process, especially for vehicle production. Ducker worldwide estimates that stamped steel parts comprise 40\% of an average light-duty vehicle on the North American market in 2015, while stamped aluminium parts account for another 2\% (Ducker Worldwide 2017 \cite{Dai2017}). It can be observed from Figure \ref{fig:stampingenergy}, that stamping by hydraulic presses is more energy-intensive than by mechanical presses \cite{Dai2017}.

\begin{figure}[h!]
	\centering
	\includegraphics[width=1\linewidth]{Figure/report-energy-stamping.jpg}
	\caption{Reported Energy Consumption for Stamping}
	\label{fig:stampingenergy}
\end{figure}

\textbf{Hybrid moudling}

Hybrid processes are an efficient solution for shortening process chains and improving performance. The merging of forming and moulding processes into a single step process allows generating potential cost-savings, especially for medium to large production volumes. Based on the technological comparison between the conventional and the new process chain, the determination of the energy consumption and the detected impacts concerning the process efficiency are presented, showing that approximately 20\% less energy is required due to only using one tempered tooling and reduction of handling operation. In total, the conventional process time is 18\% longer and requires 20\% more energy \cite{Landgrebe2016}.

\subsection{Subtractive Processes}

Machining is any of various processes in which a piece of raw material is cut into a desired final shape and size by a controlled material-removal process. The processes that have this common theme, controlled material removal, are today collectively known as subtractive manufacturing. The three principal machining processes are classified as turning, drilling and milling. 

To perform the operation, relative motion is required between the tool and the work. This relative motion is achieved in most machining operation by means of a primary motion, called "cutting speed" and a secondary motion called "feed". The shape of the tool and its penetration into the work surface, combined with these motions, produce the desired shape of the resulting work surface. A commonly used processed is CNC (computer numerical control) machining, which is a manufacturing process in which pre-programmed computer software dictates the movement of factory tools and machinery.

\begin{figure}[h!]
	\centering
	\includegraphics[width=0.8\linewidth]{Figure/subtractive-process.jpg}
	\caption{Scheme subtractive process}
	\label{fig:subtractiveprocess}
\end{figure}


\subsubsection{Possible improvements}

Machining is one of the major activities in manufacturing industries and is responsible for a significant portion of the total consumed energy in this sector. Performing machining processes with better energy efficiency will, therefore, significantly reduce the total industrial consumption of energy.

\textbf{Tool impact on energy consumption}

Improvements in energy efficiency of machine tools are possible by using Diamond-Like Carbon (DLC) deposited tools. This modification in tool-chip contact mechanics was shown to be able to reduce the cutting power consumption of the machine by 36\%.

A study by Neugebauer et al. shows that the share of ancillary devices and supporting systems in the total energy consumption of the machine tools typically increases when the size of the machine tool grows. Therefore, choosing smaller machine tools in planning and scheduling phases can potentially reduce the energy consumption for the same machining job \cite{Newman2012}.

\textbf{Process Parameters}

Enhancing the tool-chip contact conditions for reducing energy consumption has been investigated, but other research shows that the material cutting process itself uses only about 20\% of the total energy consumed by the machine during the material removal process. The rest of the energy is consumed in other parts inside the machine; the controller, fluid pump, fan and other ancillary devices are responsible for a part of the total energy consumed by the machine. The gap between the energy consumed by the machine tool and the actual energy required for material removal is the total potential for saving energy in the CNC metal Cutting process.

As mentioned above, the energy is used not only in driving the mechanical elements of the machine directly related to the cutting process, but also to run auxiliary devices such as coolant dispensing mechanisms and electronics. In order to make the metal cutting solution more energy efficient, it is possible to make more energy conscious use of the current available resources by considering the energy use as a criterion in process planning. Experiments show that the energy consumption of interchangeable machining processes can differ significantly, by at least 6\% of the total energy consumption of the machine in low loads and is likely to grow to 40\% at higher loads \cite{Newman2012}. To illustrate this phenomenon, Diaz et al. \cite{Diaz2011} showed that the energy consumption for drilling and face/end milling can be reduced by setting the cutting conditions (cutting speed, feed rate and cutting depth) high, thereby shortening the machining time, yet within a value range which does not compromise tool life and surface finish \cite{Duflou2012}. Figure \ref{fig:specificenergymilling} shows the specific energy demand for milling process as function of the material removal rate \cite{Duflou2012}.

\begin{figure}[h!]
	\centering
	\includegraphics[width=0.8\linewidth]{Figure/specific-energy-milling.jpg}
	\caption{Specific energy demand for milling processes as function of the material removal rate}
	\label{fig:specificenergymilling}
\end{figure}

For deep hole machining, the power consumption can be reduced with an adaptive pecking cycle, which executes pecking as needed by sensing cutting load. Eventually, synchronization of the spindle acceleration/deceleration with the feed system during a rapid traverse stage can reduce the energy consumption by 10\% \cite{Duflou2012}.

\textbf{Auxiliary devices optimization}

Another study realized by Pusavec et al. present cryogenic and high-pressure jet assisted machining (HPJAM), using liquid nitrogen as a coolant, as viable machining technologies offering a cost-effective route to improve economic and environmental performance in comparison to flood cooling in conventional machining \cite{Duflou2012}.


\subsection{Additive processes}
%------ definition of addtive processes

Additive manufacturing (AM) processes, generally known as 3-D printing (3DP), basically forms a structure using layer manufacturing (LM) technologies. Because AM directly builds the part layer upon layer, it offers high flexibility in prototyping and manufacturing. Figure \ref{fig:additiveprocess} shows the basic method how an additive process works.

\begin{figure}[h!]
	\centering
	\includegraphics[width=0.9\linewidth]{Figure/additive-process.jpg}
	\caption{Scheme additive process \cite{Yoon2014}}
	\label{fig:additiveprocess}
\end{figure}

%------ specific model for additive processes 

Watson J. et al. \cite{Watson2018} introduces a advances model for determining the energy. All the different terms can be appointed to the respective energy category mentioned in section General machine. The symbols definitions are provided in table \ref{table:terms} and table \ref{table:symbols}. $E_{ID}$ and $ E_{IMA} $ represent the idle energy associated with the deposition process and the finish machining of the deposited structure, respectively.

\begin{equation} \label{eq:AdditiveModel}
E_{D} = \rho \alpha V_T E_F +  \rho \alpha V_T E_T x_F + \alpha V_T E_{VA} + f {\alpha} V_T E_{VS} +  f \alpha V_T \rho E_T x_S + E_{ID} + E_{IMA}
\end{equation}

\begin{table}[H]
\centering
\caption{Definition of symbols \cite{Watson2018}}
\label{table:symbols}
\begin{tabular}{l p{9cm}}
\hline
$ V_M $ & Volume of deposited material   \\
$ V_T $ & Volume defined by part envelope  \\
$ \alpha $ & Fraction of part envelope containing solid material e the ''solid-to-envelop ratia'' ($V_M$/$V_T$)  \\
$ E_{VA} $  & Energy/unit volume of material added   \\
$ E_{VS}$  & Energy/unit volume of material subtracted   \\
$ f $ & Fraction of deposited material removed by machining   \\
$ \rho  $ & Density of material  \\
$ E_T $ & Energy/kg-km for transporting material  \\
$ E_F  $ & Energy/kg for production of feedstock \\
$ E_B  $ & Energy/kg for billet production \\
$ x_F  $ & Distance that feedstock is transported \\
$ x_B  $ & Distance that billet or plate is transported \\
$ x_S  $ & Distance that scrap is transported for recycling \\ \hline
\end{tabular}
\end{table}
\begin{table}[H]
\centering
\caption{Definition of terms \cite{Watson2018}}
\label{table:terms}
\begin{tabular}{l p{9cm}}
\hline
$\alpha V_T E_{VD} $ & Energy for deposition  \\
$ f {\alpha} V_T E_{VM} $ &  Energy for final machining  \\
$ \rho \alpha V_T E_T x_F $ &  Energy for transport of feedstock  \\
$ \rho \alpha V_T E_F $ &  Energy for production of feedstock \\
$ f \alpha V_T \rho E_T x_S $ &  Energy for transport of swarf  from finish machining of deposited material to recycling \\ \hline
\end{tabular}
\end{table}


%----- energy distribution additive specific case \newline
Figure \ref{fig:powerdistribution}  shows a lot of information about an additive process called SLS. This information is the determined with the aid of experiments. It helps to setup a model like previous mentioned. There are lots of different AM, in the category plastic methods you have processes like Stereolithography apparatus (SLA) and Selective Laser Sintering (SLS), Direct Metal Laser Sintering (DMLS) and Selective Laser Melting (SLM) are processes from the metal methods. All these different types of machine are carefully measured and added to LCI databases.

\begin{figure}[h!]
	\centering
	\includegraphics[width=0.5\linewidth]{Figure/Power-distribution-sls.jpg}
	\caption{Power distribution of selective laser sintering (SLS) process \cite{Yoon2014}}
	\label{fig:powerdistribution}
\end{figure}

\subsubsection{Possible improvements}

Most of the studies out there analyze the energy consumption but they don't propose a method on how to improve a AM machine. Frazier W. \cite{Frazier2014} gives an overview of some studies analyzing metal additive manufacturing. There are some works like the study of Peng T. \cite{Peng2016} who proposed a process model for energy analysis that they further use to optimize the 3D printed design. As you can see in Figure \ref{fig:energyanalysismodelAM} they use a much more states then shown in the general machine states.
\begin{figure}[h!]
	\centering
	\includegraphics[width=0.5\linewidth]{Figure/model-energystate-AM.JPG}
	\caption{A process model for energy analysis \cite{Peng2016}}
	\label{fig:energyanalysismodelAM}
\end{figure}



\subsection{Industrial robots}

%--------Short intro

This is because the energy consumption of IR is approximately 8 \% of the total electrical energy consumed in production processes \cite{Engelmann2009}. Therefore, a reduction in the energy consumption of IR is very important in order to improve manufacturing systems efficiency.

The basis for modeling energy consumption of an industrial robot rests on the simple multiplication of torque and speed. The implementation of this formula is not as simple. For an accurate energy calculation the power losses should also be taken in to acount. The power Losses consist of a mechanical part and in electrical part. Paryanto B. et al. \cite{Paryanto2015} uses software called Catia Systems Engineering to calculate the torque depending on the acceleration and mass of the different components.

\begin{equation} \label{eq:PowerIR}
\begin{split}
 P = \sum_{i=1}^{n} T_i \omega_i \frac{1}{\prod_{i=1}^{n} \eta_{m,i} \eta_{e,i}}   \\
 W  = \int_{0}^{t} P dt
\end{split}
\end{equation}

In comparison with the previous processes the industrial robot doesn't produce any parts. It is component who connect different machines through moving parts form one place to another. Being different in the aspect of not producing a part it still can be model through the general model.

An analysis on a industrial robot is performed by \cite{Chemnitz2011}. They come to a conclusion that the consumed energy is quadratic polynomial dependent over the time of motion. These dependence can easely be used in black blox calculations of a industrial robot

\subsubsection{Possible improvements}

Meike D. et al. \cite{Meike2011} gives an overview of all the possible improvment that are possible with a industrial robot. A lot of these methods are similar to the methodes for optimizing a general machine.

\begin{itemize}  
\item \textit{Usage strategy}; this includes the right choice of robot and Various stand-by modes during the production free-time.
\item \textit{Intelligent mechanical brake management}, with this method the energy consumption can be improved by controlling the brake due to the fact that when the breaks active the robot doesn't need any power.
\item \textit{Tool weight reduction}, the energy is dependent on the torque which is also dependent on the inertia of the tool. By reducing the weight the energy is also reduced.
\item \textit{Reuse of the kinetic energy}, a motion consist of acceleration and decelleration, the energy from a decelerating motor can be used in a accelerating motor this between motors of one robot or between several robots.
\end{itemize}

The study of  M. Pellicciari \cite{Pellicciari2013} uses constant time scaling to improve the energy consumption, starting from pre-scheduled trajectories compatible with the actuation limits. They change the task execution time with the aid of a scaling factor. In the case study they performed, a clear optimum point of execution time is visible for a minimum energy consumption.







%----------------------------------------------------------------------------------------
