
% License:
% CC BY-NC-SA 3.0 (http://creativecommons.org/licenses/by-nc-sa/3.0/)
%
%%%%%%%%%%%%%%%%%%%%%%%%%%%%%%%%%%%%%%%%%

%----------------------------------------------------------------------------------------
%	PACKAGES AND OTHER DOCUMENT CONFIGURATIONS
%----------------------------------------------------------------------------------------

\documentclass[paper=a4, fontsize=11pt]{scrartcl} % A4 paper and 11pt font size

\usepackage[T1]{fontenc} % Use 8-bit encoding that has 256 glyphs
\usepackage{fourier} % Use the Adobe Utopia font for the document - comment this line to return to the LaTeX default
\usepackage[english]{babel} % English language/hyphenation
\usepackage{amsmath,amsfonts,amsthm} % Math packages
\usepackage{lipsum} % Used for inserting dummy 'Lorem ipsum' text into the template

\usepackage{caption}
\usepackage{subcaption}
\usepackage{graphicx}

\usepackage{float}

\usepackage{blindtext} %for enumarations

\usepackage[]{hyperref}  %link collor

%talbe layout to the right
%\usepackage[labelfont=bf]{caption}
%\captionsetup[table]{labelsep=space,justification=raggedright,singlelinecheck=off}
%\captionsetup[figure]{labelsep=quad}

\usepackage{sectsty} % Allows customizing section commands
\allsectionsfont{\centering \normalfont\scshape} % Make all sections centered, the default font and small caps

\usepackage{fancyhdr} % Custom headers and footers
\pagestyle{fancyplain} % Makes all pages in the document conform to the custom headers and footers
\fancyhead{} % No page header - if you want one, create it in the same way as the footers below
\fancyfoot[L]{} % Empty left footer
\fancyfoot[C]{} % Empty center footer
\fancyfoot[R]{\thepage} % Page numbering for right footer
\renewcommand{\headrulewidth}{0pt} % Remove header underlines
\renewcommand{\footrulewidth}{0pt} % Remove footer underlines
\setlength{\headheight}{13.6pt} % Customize the height of the header

\numberwithin{equation}{section} % Number equations within sections (i.e. 1.1, 1.2, 2.1, 2.2 instead of 1, 2, 3, 4)
\numberwithin{figure}{section} % Number figures within sections (i.e. 1.1, 1.2, 2.1, 2.2 instead of 1, 2, 3, 4)
\numberwithin{table}{section} % Number tables within sections (i.e. 1.1, 1.2, 2.1, 2.2 instead of 1, 2, 3, 4)

%\setlength\parindent{0pt} % Removes all indentation from paragraphs - comment this line for an assignment with lots of text


\setlength\parskip{4pt}

%----------------------------------------------------------------------------------------
%	TITLE SECTION
%----------------------------------------------------------------------------------------

\newcommand{\horrule}[1]{\rule{\linewidth}{#1}} % Create horizontal rule command with 1 argument of height

\title{	
\normalfont \normalsize 
\textsc{Shanghai Jiao Tong University} \\ [25pt] % Your university, school and/or department name(s)
\horrule{0.5pt} \\[0.4cm] % Thin top horizontal rule
\huge  Literature review for energy consumption in manufacturing industry from machine to factory level \\ % The assignment title
\horrule{2pt} \\[0.5cm] % Thick bottom horizontal rule
}

\author{  Wouter Vanoverberghe,Chang Yuan ...} % Your name

\date{\normalsize\today} % Today's date or a custom date

\begin{document}
%\nocite{*}
\maketitle % Print the title

\newpage
\begin{abstract}

{\noindent{\Large Abstract} }\\

Energy consumption is one of the most critical issues in the manufacturing industry. The modeling, analysis and improvement of energy cost and consumption in multistage production system have been widely studied in many research works. To summarize the latest development of research of energy consumption, a large amount of research work have been investigated. The review work includes effect of design for the energy consumption and interactions between many aspects related to industry. This research research combines energy systems from microscopic to macroscopic, which includes machine, manufacturing line and factory level. 
This document begins with a review of energy consumption on machine level. Including the schema for machine states and transition of energy, the process model for energy analysis and improvement methods. These topics are further discussed in detail for different processes, eg. forming process, additive processes, etc. In the next part of the review the researches on energy consumption for the multi-machine/manufacturing line level are introduced. Include the define of manufacturing line level, the logical benchmarking of manufacturing lines, and the utilization of energy flows. At last, the detailed way for the factory level energy management are studied. Including the factory energy management system and the method of reduce energy consumption.  \\


\textbf{Key words:} review, Energy Efficiency, Manufacturing Processes, Machine Optimization, Sustainable manufacturing, Research activities
    
\end{abstract}

\newpage
\tableofcontents

%----------------------------------------------------------------------------------------
%	Section 1
%----------------------------------------------------------------------------------------

\input{1-Section.tex}

%----------------------------------------------------------------------------------------
%	Section 2
%----------------------------------------------------------------------------------------

\newpage
\section{Machine level}
\label{chapter2}
\textit{}
%---- the scope of this chapter}}

The discrete part manufacturing processes will be discussed in detail. It is defined as production processes in which the output can be identified and is measurable in distinct units\cite{Duflou2012}. Manufacturing processes are already widely described in previous literature. Unit process boundaries are characterized through Figure \ref{fig:BoundariesProcess}. A unit process can be described as an individual machine tool.


%---- Define machine level 

A sequence of unit processes make up a general production process, a unit process should be clearly defined where it starts and where the process ends, or in other words define all the boundaries to the environment and other unit processes. Machine level is not the lowest level, also sub-unit level studies need to be performed\cite{Kellens2012}.

\begin{figure}[h!]
	\centering
	\includegraphics[width=0.8\linewidth]{Figure/Unit-process.JPG}
	\caption{System boundaries of a unit process \cite{Kellens2012}}
	\label{fig:BoundariesProcess}
\end{figure}

In chapter \ref{chapter2} energy consumption of machine tools during there use stage is discussed. Various processes are analyzed and collected in a life cycle inventory (LCI) databases. Different industries or entities make a LCI. The databases from LCI can give good descriptions for material production and manufacturing processes but sometimes lack in overall environmental impact\cite{Duflou2012}. Examples of LCI databases are; Ecoinvent database \cite{Frischknecht2007} European Aluminium Association \cite{EuropeanAluminiumAssociation2018} and Plastics Europe \cite{EcoProfiles}. There are a lot of different manufacturing processes investigated by the LCI databases. These databases can cover a full process taxonomy like DIN8580 \cite{DIN2003}. But LCI databases have several shortcomings that improve every year, e.g. it being limited to primary material production or only limit to more conventional processes. 

%-------    CO2PE!-initiative

Performing life cycle analysis (LCA) is necessary for the further growth of energy efficient manufacturing. This growth is written down in an initiative to better coordinate all the research. The CO2PE!-Initiative \cite{Kellens2012} has as the objective to coordinate international efforts aiming to document and analyse the environmental impacts of a wide range of current and emerging manufacturing processes, and to provide guidelines to reduce these impacts.

The databases provide information about manufacturing processes which can be used to calculate a complete production line to discuss total energy consumption and so on. Only basic energy energy parameters are provided by these databases, the concept is explained in the General machine section. Besides a basic model for energy consumption specific processes  also need to examine, these include forming processes, substractive processes, additive processes and industrial robots.

\subsection{General machine}

The energy calculation for life cycle inventory is basically similar for all the production processes. The consumed energy of a machine can be split up in to three parts according to Abele et al.\cite{Abele2005}.
\begin{itemize}  
\item $E_{th}$, the active energy 
\item $E_{additional}$, the additional energy requirement of the machine 
\item $E_{periphery}$, the energy demand of the process-periphery 
\end{itemize}

The active energy $E_{th}$ incorporates the theoretical energy needed to perform the change of shape. It represents the minimum energy consumption of a machine. 

The transformation energy is only a fraction of the total energy used by the machine. Additional energy is calculated with equation \ref{eq:Eadditional}. It includes the basic power $P_{basic}$ and the average idle power $P_{idle}$ with respectively the basic time $t_{b}$and the work-piece utilization time $t_{U}$.

\begin{equation} \label{eq:Eadditional}
E_{additional} = P_{idle} \cdot t_b + P_{basic} \cdot (t_{U} - t_{b})
\end{equation}

Also the process periphery is used in the energy calculations. It is an summation of all the single electricity consumers multiplied by the work piece utilization time.

\begin{equation} \label{eq:Eperiphery}
E_{periphery,el} = \sum_{i} P_{i} \cdot t_U
\end{equation}

In most studies a general concept based on machines states is used, see Figure \ref{fig:energymachine}. Schrouf et al.\cite{Shrouf2014} and Bajpai et al.\cite{Bajpai2018} are two examples who use this concept. These studies all perform energy consumption calculations on production line level, which is explained in chapter \ref{chapter3}. Each state has it's own time value depending on the production line and a power value.

\begin{figure}[h!]
	\centering
	\includegraphics[width=0.6\linewidth]{Figure/energymachine.JPG}
	\caption{A schema for machine states and transition \cite{Shrouf2014}}
	\label{fig:energymachine}
\end{figure}

%---- How to detirmine the LCI


\subsection{Energy reduction strategies}

Optimizing a single machine is done by the machine tool manufacturers. They optimize the machine tool design, this is different from process planners who optimize the entire process through the process parameters\cite{Duflou2012}. Energy reduction begins with a good selection of the needed tools or also considering substituting a tool when necessary.\newline

\textbf{Optimized machine tool design}

\begin{itemize}  
\item \textit{More efficient machine tool components}, improving a machine can obliviously be done by improving the components efficiency. Examples are more efficient drives, pumps, spindles, etc.
\item \textit{Technological}, instead of just improving the components efficiency. Adopting a new technology could lead to an significant gain. For example replacing the conventional C02-lasers with a new generation of diode lasers.
\item \textit{Waste recovery within a machine tool}, energy improvement can also be achieved by recovering energy to reuse. Diaz et al. \cite{Diaz2009} uses energy recovery in a milling machine through kinetic energy recovery system (KERS). They improved the energy consumption up to 25\%.
\end{itemize}


\textbf{Effects of optimized process control}
\begin{itemize}  
\item \textit{Selective actuation of non-continuously required devices}, selectively shutting down devices of which the functionality is not required in specific operations. Example: switching to a less energy consuming stand-by level.
\item \textit{Reducing idle production times}, the second parameter in energy demand is time. Reducing the unproductive idle time during a manufacturing can lead to an overall increase energy efficiency.
\item \textit{Optimized process parameters}, a significant decrease in energy consumption can be obtained through the right selection of process parameters. Examples is increasing cutting speed when milling, thereby shortening the machining time and reducing the energy. 
\item \textit{Energy and resource efficient process modeling and planning}, Various models for energy, resource efficient process modeling, planning and scheduling have been presented in literature. 
\end{itemize}

\textbf{Effects of process/machine tool selection}

Selecting a certain tool is always done by an comparison. Energy is not the only selection parameter for comparing machines but is certainly very important. Different studies have already been performed on comparing different types of machines. Yoon H. et al. \cite{Yoon2014} did a comparison of energy consumption between bulk forming, subtractive, and additive processes. Others like Watson J. et al. \cite{Watson2018} made a decision-support model for selecting additive manufacturing versus subtractive manufacturing based on energy consumption.

\subsection{Forming Processes}

Forming processes are manufacturing processes which make use of suitable stresses (like compression, tension, shear or combined stresses) which cause plastic deformation of the materials to produce required shapes. 

\begin{figure}[H]
	\centering
	\includegraphics[width=0.8\linewidth]{Figure/formingprocess.jpg}
	\caption{overall forming process}
	\label{fig:formingprocess}
\end{figure}

The current technology to produce sheet metal parts in high quantities is primarily based on the use of mechanical forming presses. A common design is the eccentric drive in addition to a flywheel which is driven by an asynchronous electric motor. The flywheel is coupled with the eccentric shaft via a clutch and the ram is mechanically connected to the eccentric shaft by means of conrods. The eccentric bearing of the conrods leads to a sinusoidal motion of the ram as shown in Figure \ref{fig:formingdesign} \cite{Behrens2016}.

\begin{figure}[H]
	\centering
	\includegraphics[width=0.8\linewidth]{Figure/ramdesign.jpg}
	\caption{Common design of forming part of the process}
	\label{fig:formingdesign}
\end{figure}

The ram's stroke rate is regulated by the rotational motor speed with a frequency converter. In order to realize a high output rate automated press lines are used, in which the sheet metal is fed from a coil by means of sheet metal feeders.

\subsubsection{Possible improvements}

In the forming processes the forming force is very high. For instance, in the automobile panel forming process the forming force is nearly up to 9000kN, but the machine energy efficiency of the drawing process is low to 7\%; the main reason is the mismatch between the output power of the equipment and the demanded power of the process, which leads to large energy consumption. Therefore, there is a desire for models or tools for energy consumption reduction and energy efficiency improvement \cite{Gao2018}.

\textbf{Metal feeding}

New energy-efficient drive concepts for forming presses and sheet metal feeding systems are developed. The novel press drive is based on a power-split design, which allows a variable ram-kinematics with reduced total costs of ownership in comparison to conventional servo presses. The new feeding concept will be able to realize the contactless feed of electrically conductive sheet metals by means of electromagnetic forces. Since only the sheet metal must be accelerated, the energy efficiency and feeding rate can be increased significantly \cite{Behrens2016}.

\textbf{Stamping}

Stamping is an important sheet metal forming process, especially for vehicle production. Ducker worldwide estimates that stamped steel parts comprise 40\% of an average light-duty vehicle on the North American market in 2015, while stamped aluminium parts account for another 2\% (Ducker Worldwide 2017 \cite{Dai2017}). It can be observed from Figure \ref{fig:stampingenergy}, that stamping by hydraulic presses is more energy-intensive than by mechanical presses \cite{Dai2017}.

\begin{figure}[h!]
	\centering
	\includegraphics[width=1\linewidth]{Figure/report-energy-stamping.jpg}
	\caption{Reported Energy Consumption for Stamping}
	\label{fig:stampingenergy}
\end{figure}

\textbf{Hybrid moudling}

Hybrid processes are an efficient solution for shortening process chains and improving performance. The merging of forming and moulding processes into a single step process allows generating potential cost-savings, especially for medium to large production volumes. Based on the technological comparison between the conventional and the new process chain, the determination of the energy consumption and the detected impacts concerning the process efficiency are presented, showing that approximately 20\% less energy is required due to only using one tempered tooling and reduction of handling operation. In total, the conventional process time is 18\% longer and requires 20\% more energy \cite{Landgrebe2016}.

\subsection{Subtractive Processes}

Machining is any of various processes in which a piece of raw material is cut into a desired final shape and size by a controlled material-removal process. The processes that have this common theme, controlled material removal, are today collectively known as subtractive manufacturing. The three principal machining processes are classified as turning, drilling and milling. 

To perform the operation, relative motion is required between the tool and the work. This relative motion is achieved in most machining operation by means of a primary motion, called "cutting speed" and a secondary motion called "feed". The shape of the tool and its penetration into the work surface, combined with these motions, produce the desired shape of the resulting work surface. A commonly used processed is CNC (computer numerical control) machining, which is a manufacturing process in which pre-programmed computer software dictates the movement of factory tools and machinery.

\begin{figure}[h!]
	\centering
	\includegraphics[width=0.8\linewidth]{Figure/subtractive-process.jpg}
	\caption{Scheme subtractive process}
	\label{fig:subtractiveprocess}
\end{figure}


\subsubsection{Possible improvements}

Machining is one of the major activities in manufacturing industries and is responsible for a significant portion of the total consumed energy in this sector. Performing machining processes with better energy efficiency will, therefore, significantly reduce the total industrial consumption of energy.

\textbf{Tool impact on energy consumption}

Improvements in energy efficiency of machine tools are possible by using Diamond-Like Carbon (DLC) deposited tools. This modification in tool-chip contact mechanics was shown to be able to reduce the cutting power consumption of the machine by 36\%.

A study by Neugebauer et al. shows that the share of ancillary devices and supporting systems in the total energy consumption of the machine tools typically increases when the size of the machine tool grows. Therefore, choosing smaller machine tools in planning and scheduling phases can potentially reduce the energy consumption for the same machining job \cite{Newman2012}.

\textbf{Process Parameters}

Enhancing the tool-chip contact conditions for reducing energy consumption has been investigated, but other research shows that the material cutting process itself uses only about 20\% of the total energy consumed by the machine during the material removal process. The rest of the energy is consumed in other parts inside the machine; the controller, fluid pump, fan and other ancillary devices are responsible for a part of the total energy consumed by the machine. The gap between the energy consumed by the machine tool and the actual energy required for material removal is the total potential for saving energy in the CNC metal Cutting process.

As mentioned above, the energy is used not only in driving the mechanical elements of the machine directly related to the cutting process, but also to run auxiliary devices such as coolant dispensing mechanisms and electronics. In order to make the metal cutting solution more energy efficient, it is possible to make more energy conscious use of the current available resources by considering the energy use as a criterion in process planning. Experiments show that the energy consumption of interchangeable machining processes can differ significantly, by at least 6\% of the total energy consumption of the machine in low loads and is likely to grow to 40\% at higher loads \cite{Newman2012}. To illustrate this phenomenon, Diaz et al. \cite{Diaz2011} showed that the energy consumption for drilling and face/end milling can be reduced by setting the cutting conditions (cutting speed, feed rate and cutting depth) high, thereby shortening the machining time, yet within a value range which does not compromise tool life and surface finish \cite{Duflou2012}. Figure \ref{fig:specificenergymilling} shows the specific energy demand for milling process as function of the material removal rate \cite{Duflou2012}.

\begin{figure}[h!]
	\centering
	\includegraphics[width=0.8\linewidth]{Figure/specific-energy-milling.jpg}
	\caption{Specific energy demand for milling processes as function of the material removal rate}
	\label{fig:specificenergymilling}
\end{figure}

For deep hole machining, the power consumption can be reduced with an adaptive pecking cycle, which executes pecking as needed by sensing cutting load. Eventually, synchronization of the spindle acceleration/deceleration with the feed system during a rapid traverse stage can reduce the energy consumption by 10\% \cite{Duflou2012}.

\textbf{Auxiliary devices optimization}

Another study realized by Pusavec et al. present cryogenic and high-pressure jet assisted machining (HPJAM), using liquid nitrogen as a coolant, as viable machining technologies offering a cost-effective route to improve economic and environmental performance in comparison to flood cooling in conventional machining \cite{Duflou2012}.


\subsection{Additive processes}
%------ definition of addtive processes

Additive manufacturing (AM) processes, generally known as 3-D printing (3DP), basically forms a structure using layer manufacturing (LM) technologies. Because AM directly builds the part layer upon layer, it offers high flexibility in prototyping and manufacturing. Figure \ref{fig:additiveprocess} shows the basic method how an additive process works.

\begin{figure}[h!]
	\centering
	\includegraphics[width=0.9\linewidth]{Figure/additive-process.jpg}
	\caption{Scheme additive process \cite{Yoon2014}}
	\label{fig:additiveprocess}
\end{figure}

%------ specific model for additive processes 

Watson J. et al. \cite{Watson2018} introduces a advances model for determining the energy. All the different terms can be appointed to the respective energy category mentioned in section General machine. The symbols definitions are provided in table \ref{table:terms} and table \ref{table:symbols}. $E_{ID}$ and $ E_{IMA} $ represent the idle energy associated with the deposition process and the finish machining of the deposited structure, respectively.

\begin{equation} \label{eq:AdditiveModel}
E_{D} = \rho \alpha V_T E_F +  \rho \alpha V_T E_T x_F + \alpha V_T E_{VA} + f {\alpha} V_T E_{VS} +  f \alpha V_T \rho E_T x_S + E_{ID} + E_{IMA}
\end{equation}

\begin{table}[H]
\centering
\caption{Definition of symbols \cite{Watson2018}}
\label{table:symbols}
\begin{tabular}{l p{9cm}}
\hline
$ V_M $ & Volume of deposited material   \\
$ V_T $ & Volume defined by part envelope  \\
$ \alpha $ & Fraction of part envelope containing solid material e the ''solid-to-envelop ratia'' ($V_M$/$V_T$)  \\
$ E_{VA} $  & Energy/unit volume of material added   \\
$ E_{VS}$  & Energy/unit volume of material subtracted   \\
$ f $ & Fraction of deposited material removed by machining   \\
$ \rho  $ & Density of material  \\
$ E_T $ & Energy/kg-km for transporting material  \\
$ E_F  $ & Energy/kg for production of feedstock \\
$ E_B  $ & Energy/kg for billet production \\
$ x_F  $ & Distance that feedstock is transported \\
$ x_B  $ & Distance that billet or plate is transported \\
$ x_S  $ & Distance that scrap is transported for recycling \\ \hline
\end{tabular}
\end{table}
\begin{table}[H]
\centering
\caption{Definition of terms \cite{Watson2018}}
\label{table:terms}
\begin{tabular}{l p{9cm}}
\hline
$\alpha V_T E_{VD} $ & Energy for deposition  \\
$ f {\alpha} V_T E_{VM} $ &  Energy for final machining  \\
$ \rho \alpha V_T E_T x_F $ &  Energy for transport of feedstock  \\
$ \rho \alpha V_T E_F $ &  Energy for production of feedstock \\
$ f \alpha V_T \rho E_T x_S $ &  Energy for transport of swarf  from finish machining of deposited material to recycling \\ \hline
\end{tabular}
\end{table}


%----- energy distribution additive specific case \newline
Figure \ref{fig:powerdistribution}  shows a lot of information about an additive process called SLS. This information is the determined with the aid of experiments. It helps to setup a model like previous mentioned. There are lots of different AM, in the category plastic methods you have processes like Stereolithography apparatus (SLA) and Selective Laser Sintering (SLS), Direct Metal Laser Sintering (DMLS) and Selective Laser Melting (SLM) are processes from the metal methods. All these different types of machine are carefully measured and added to LCI databases.

\begin{figure}[h!]
	\centering
	\includegraphics[width=0.5\linewidth]{Figure/Power-distribution-sls.jpg}
	\caption{Power distribution of selective laser sintering (SLS) process \cite{Yoon2014}}
	\label{fig:powerdistribution}
\end{figure}

\subsubsection{Possible improvements}

Most of the studies out there analyze the energy consumption but they don't propose a method on how to improve a AM machine. Frazier W. \cite{Frazier2014} gives an overview of some studies analyzing metal additive manufacturing. There are some works like the study of Peng T. \cite{Peng2016} who proposed a process model for energy analysis that they further use to optimize the 3D printed design. As you can see in Figure \ref{fig:energyanalysismodelAM} they use a much more states then shown in the general machine states.
\begin{figure}[h!]
	\centering
	\includegraphics[width=0.5\linewidth]{Figure/model-energystate-AM.JPG}
	\caption{A process model for energy analysis \cite{Peng2016}}
	\label{fig:energyanalysismodelAM}
\end{figure}



\subsection{Industrial robots}

%--------Short intro

This is because the energy consumption of IR is approximately 8 \% of the total electrical energy consumed in production processes \cite{Engelmann2009}. Therefore, a reduction in the energy consumption of IR is very important in order to improve manufacturing systems efficiency.

The basis for modeling energy consumption of an industrial robot rests on the simple multiplication of torque and speed. The implementation of this formula is not as simple. For an accurate energy calculation the power losses should also be taken in to acount. The power Losses consist of a mechanical part and in electrical part. Paryanto B. et al. \cite{Paryanto2015} uses software called Catia Systems Engineering to calculate the torque depending on the acceleration and mass of the different components.

\begin{equation} \label{eq:PowerIR}
\begin{split}
 P = \sum_{i=1}^{n} T_i \omega_i \frac{1}{\prod_{i=1}^{n} \eta_{m,i} \eta_{e,i}}   \\
 W  = \int_{0}^{t} P dt
\end{split}
\end{equation}

In comparison with the previous processes the industrial robot doesn't produce any parts. It is component who connect different machines through moving parts form one place to another. Being different in the aspect of not producing a part it still can be model through the general model.

An analysis on a industrial robot is performed by \cite{Chemnitz2011}. They come to a conclusion that the consumed energy is quadratic polynomial dependent over the time of motion. These dependence can easely be used in black blox calculations of a industrial robot

\subsubsection{Possible improvements}

Meike D. et al. \cite{Meike2011} gives an overview of all the possible improvment that are possible with a industrial robot. A lot of these methods are similar to the methodes for optimizing a general machine.

\begin{itemize}  
\item \textit{Usage strategy}; this includes the right choice of robot and Various stand-by modes during the production free-time.
\item \textit{Intelligent mechanical brake management}, with this method the energy consumption can be improved by controlling the brake due to the fact that when the breaks active the robot doesn't need any power.
\item \textit{Tool weight reduction}, the energy is dependent on the torque which is also dependent on the inertia of the tool. By reducing the weight the energy is also reduced.
\item \textit{Reuse of the kinetic energy}, a motion consist of acceleration and decelleration, the energy from a decelerating motor can be used in a accelerating motor this between motors of one robot or between several robots.
\end{itemize}

The study of  M. Pellicciari \cite{Pellicciari2013} uses constant time scaling to improve the energy consumption, starting from pre-scheduled trajectories compatible with the actuation limits. They change the task execution time with the aid of a scaling factor. In the case study they performed, a clear optimum point of execution time is visible for a minimum energy consumption.







%----------------------------------------------------------------------------------------



\newpage
\section{Manufacturing line or multi-machine level}
\label{chapter3}

\textit{Person in Charge; Jin Chenghao and Huang Xinpei}

%---------The scope of this chapter

In this chapter,the researches on energy consumption on the multi-machine/manufacturing line level are introduced.Possible interactions and synergies between different machine tools are considered. Multi-machine ‘ecosystems’ can allow reuse of energy and material flows through proper planning and control.

%---------Define manufacturing line level 

A manufacturing line is a combination of different production processes and typically is composed of diverse machines for processing or transportation as well as personnel. All these production factors are being planned and controlled by a production management system. A multi-machine process boundaries are characterized through Figure \ref{fig:line-boundaries}. 

\begin{figure}[h!]
	\centering
	\includegraphics[width=0.8\linewidth]{Figure/line-boundaries.jpg}
	\caption{System boundaries of a unit process }
	\label{fig:line-boundaries}
\end{figure}

Majority of the energy management literature found in the manufacturing field is at the machine level and focuses on enhancing energy efficiency by selecting optimally either cutting conditions It was discovered that there can be an 80\% reduction in energy consumption if instead of leaving non-bottleneck machines idle, these machines are turned off until needed \cite{Gilles2007}. There have been research efforts that discovered that 85\% of energy in a manufacturing environment is utilized for functions not related to the production of parts \cite{Gutowski2005}. This suggests that with the development of integrated control efforts of system level operations there are many energy savings opportunities for the production line.There has been work into production scheduling that takes into account energy and environmental factors \cite{Fang2011}.There has been a work that proposes a modeling method of task-oriented energy consumption for machining manufacturing system. The energy consumption characteristics driven by task flow in machining manufacturing system are analyzed, which describes that energy consumption dynamically depends on the flexibility and variability of task flow in production processes\cite{He2012}.The Energy Blocks methodology for accurate energy consumption prediction is introduced,which is based on the representation of production operations as segments of specific energy consumption for each operating state of the production equipment and modelling any process chain is possible by arranging the segments according to the production programme\cite{Weinert2011}.There have been some researches on improving energy efficiency in Bernoulli serial lines like Figure \ref{fig:Bernoulli-serial-lines} and a mathematical way of reducing energy cost while maintaining desired production rate is introduced\cite{Wen2016}. 

\begin{figure}[h!]
	\centering
	\includegraphics[width=0.6\linewidth]{Figure/Bernoulli-serial-lines.jpg}
	\caption{Bernoulli serial lines}
	\label{fig:Bernoulli-serial-lines}
\end{figure}

There are also researches that use energy value stream mapping in analyzing the value-added vs. non-value-added energy use in machining cycles\cite{Muller2014}.The foundation for description, acquisition and analysis of all energetic flows into and out of a multi-machine ecosystem are the energy,exergy and entropy concepts.The energy only serves as a carrier of quality,and quality of energy or ’work potential’ can be described by the properties exergy and entropy\cite{Bejan2002}.

\subsection{Process chain design and control}

The energy consumptions are described by the energy to set up the machine and the additional energy used to process the product. That is to say. Energy is divided into two categories: the first one is related to energy required to start up the machine until it reaches ‘ready’ position, which is often fixed for a specific manufacturing process. After setting up a machine, additional requirement is proportional to its processing rate which falls into the second category.Let W$_i$ be the total electrical power consumed,W$_{o_{i}}$ represent the set-up power needed for the machine to reach ‘ready’ status, and k$_{i}p_{i}$ be the additional power, where k$_i$ is a constant, and p$_i$ describes the expected number of parts to be processed per unit of time, which is linearly characterized by the processing rate or capacity of machine i\cite{Wen2016}.

Then we obtain 
\begin{equation} \label{eq:3.1}
\begin{split}
W_{i}=W_{o_{i}}+k_{i}p_{i}
\\
\end{split}
\end{equation}

Then in a serial line with two machines, the system energy consumption within a cycle can be described as:
\begin{equation} \label{eq:3.2}
\begin{split}
E=\sum_{i=1}^{2}W_{o_{i}}+\sum_{i=1}^{2}k_{i}p_{i}
\\
\end{split}
\end{equation}

In such a Bernoulli two-machine line, let PR define the system production rate (i.e. the average number of parts produced by the last machine per unit of time). Then PR can be obtained as follows:
\begin{equation} \label{eq:3.3}
\begin{split}
PR=p_{2}[1-Q(p_{1},p_{2},N)]
\\
\end{split}
\end{equation}

Where,
\begin{equation} \label{eq:3.4}
\begin{split}
Q(p_{1},p_{2},N)=\left\{\begin{matrix}
\frac{(1-p_{1})[1-\alpha(p_{1},p_{2})]}{1-\frac{p_{1}}{p_{2}}\alpha ^{N}(p_{1},p_{2}))} ,if p_{1}\neq p_{2}
\\ 
\frac{1-p_{1}}{N+1-p_{1}}, if p_{1}=p_{2}
\end{matrix}\right.
\\
\end{split}
\end{equation}

\begin{equation} \label{eq:3.5}
\begin{split}
\alpha(p_{1},p_{2})=\frac{p_{1}(1-p_{2})}{p_{2}(1-p_{1})}
\end{split}
\end{equation}

There are two cases considered. One is with constrained workforce, i.e. limited due to capacity shortage or time restriction, while the other addresses the unconstrained case. To start, the constrained case is considered first, i.e. the workforce is limited so that optimal distribution is needed to minimize energy usage and achieve the desired production rate. Introduce workforce constraint p*=p$_1$p$_2$, then the problem is reformulated as follows:
\begin{equation} \label{eq:3.6}
\begin{split}
Min:\sum_{i=1}^{2}W_{o_{i}}+\sum_{i=1}^{2}k_{i}p_{i}
\\
s.t.\left\{\begin{matrix}
PR\geq PR_{d}\\ 
p_{1}p_{2}=p*\\ 
\quad 0<p_{1}<1 \quad 0<p_{2}<1
\end{matrix}\right.
\end{split}
\end{equation}

 Then we can draw a graph of PR-p$_{1}$ in Figure \ref{fig:PR-p1}.
 \begin{figure}[h!]
	\centering
	\includegraphics[width=0.8\linewidth]{Figure/PR-p1.png}
	\caption{PR-p1}
	\label{fig:PR-p1}
\end{figure}
 
 where critical point is P$_1$=$\sqrt{p*}$,p$_{1}'$ and p$_{1}''$ can be obtained by solving the equation:
\begin{equation} \label{eq:3.7}
\begin{split}
p_{1}^{N}(p_{1}-p*)^{N}(PR_{d}p_{1}-p*)+p*^{N+1}(1-p_{1})^{N}(p_{1}-PR_{d})=0 \quad 0<p_{1}<1
\end{split}
\end{equation}
 To minimize energy consumption while satisfying production rate constraint, the author compare the energy critical point with the production rate critical point PR and interval[p1$_{1}'$, p1$_{1}''$].
Researches have been done about the exact analysis for small systems, and the aggregation approach for medium size systems and Heuristic method for large systems\cite{Su2017}.
 
 
\subsection{Local bench marking of manufacturing lines}

Benchmarking is an important step towards forecasting energy use of prospective production lines as well as managing its use in existing lines and setting targets for reducing it. Global benchmarking in energy use can be defined as comparing energy consumption of different equipment at different plants to obtain generic ideal energy use benchmark targets as function of different operating conditions/parameters (i.e. temperature, process plans, schedules, utilization, etc.) as well as equipment characteristics (i.e.age, technology, etc.). Obtaining this extensive information is not always feasible given limited resources, measuring devices, time and information in different plants. Local energy use benchmarking sets local targets within a specific plant or manufacturing system.This has a direct impact on the evaluation of energy use efficiencies and improvement targets in many scenarios where drastic changes to current technology or manufacturing setup are not possible.The overall method of local benchmarking for energy use is composed of six steps in Figure \ref{fig:local-benchmarking}\cite{ElMaraghy2017}.

\begin{figure}[h!]
	\centering
	\includegraphics[width=0.7\linewidth]{Figure/local-benchmarking.jpg}
	\caption{local benchmarking}
	\label{fig:local-benchmarking}
\end{figure}

\subsection{Utilization of energy flows}

Thermoelectric materials are solid-state electrical, and semiconducting properties electricity or electrical power directly with fluid-based systems, such as two-used in smaller-scale applications such electrical-enclosure cooling. More widespread the intrinsic energy-conversion efficiency advancements in system architecture. In a working TE device, segments of p-type- and n-type-doped semiconductor materials, such as suitably doped bismuth telluride, are connected by shunts to form an electric circuit. The shunts are made of an excellent electrical conductor, such as copper. A voltage drives a current through the circuit, passing from one segment to another through the connecting shunts. For determining efficiency, this configuration is equivalent to the electrons passing directly from one TE material to the other. Conventional TE cooling/heating modules are constructed of pairs of TE segments, repeated about 100 times, and organized into ar rays like the one shown in Figure \ref{fig:Thermoelectric materials}\cite{Bell2008}.

\begin{figure}[H]
	\centering
	\includegraphics[width=0.7\linewidth]{Figure/Thermoelectric-materials.png}
	\caption{Thermoelectric materials}
	\label{fig:Thermoelectric materials}
\end{figure}


\newpage
\section{Factory level}
\subsection{Factory level energy consumption}
\label{chapter 4}

\textit{Person in Charge; Md Sahidul Islam and Wzheng}
%---- the scope of this chapter}}

\subsubsection{Scope}
Energy efficiency has developed into an important objective for industrial enterprises. However, there is still a need for systematic approaches to reduce energy consumption in factories. From an economic point of view, industrial enterprises have an incentive to reduce their energy consumption because of increasing energy prices, such as the European average prices for gas in industry, which rose by approximately 34\% during the last four years \cite{European2013}. 
Despite this situation, the implementation of energy efficiency measures has not met the expectations yet. The reasons for the deficits in realizing energy efficiency include lack of time, lacking transparency on energy consumption, lacking capital for investments and divided responsibilities within a company \cite{Fleiter2013}. 

Product design
Process design
Process adjustments:
Post-processing
plan and implementations


\subsubsection{define factory level}
Different tools and methods have been developed in recent years to support the systematic analysis and optimization of industrial enterprises for reducing their energy consumption. However, the existing methods mainly focus on product design, Process design, process adjustment, manufacturing processes, machine process and smart factory building design. Although these are important aspects of the energy-efficient factory, considering the interrelationships between products, processes and resources in the factory system is essential for a holistic integration of energy efficiency in the enterprise. 

\subsubsection{factory building energy consumption }
The starting point for the approach is the definition of the project task or planning situation by the factory planning participant (user input). The most important parameters to describe the task are object level, system process, part of the energy chain, energy form, planning case and user’s role. According to their background, the first four parameters are defined as technical parameters and the last two as organizational parameters. 

The object level describes the level of abstraction of the considered system (e.g. factory, building, plant area, single machine). The system process defines the process of the enterprise to which the considered system belongs to (e.g. assembly, logistics). The part of the energy chain describes whether the system performs energy generation, conversion, distribution, storage or use, since factories increasingly integrate several of these functions \cite{Muller2013a}. The energy form defines the types of resources that are used within the considered system (e.g. electricity, water). The planning case comprises the extent to which changes are possible in the system; planning a new system has the highest degrees of freedom, whereas operating the existing system equals the lowest degree of freedom. Finally, the user's role defines the perspective of the user (e.g. factory planner, worker). 

the implementing methods is the high effort for data acquisition. Therefore, an approach to reduce energy consumption within factory systems was developed that provides energy efficiency measures to factory planning participants based on qualitative data \cite{Muller2013}.

Herrmann et al.\cite{Herrmann2011} proposed a holistic definition of factory management, including technical building services as an important player \cite{Herrmann2011}. Seow and Rahimifard outlined a framework for modelling energy consumption within a manufacturing system from a product’s viewpoint, and utilized the energy data at factory and process levels \cite{Seow2011}. Mouzon et al. proposed a mathematics programming model to optimize the total energy consumption of manufacturing \cite{Gilles2007}. Clearly, a comprehensive energy consumption models at this level is still lacking. 

Manuela Krones and Egon Müller \cite{Krones2014} developed a general concept has been developed to systematically guide a factory planning participant from his or her project task to appropriate energy efficiency measures. We introduce their work in detail here. The goal is to provide suitable energy efficiency approaches in order to increase the efficiency of information gathering. The approach consists of four major steps, which are explained in the following.


When applying the method, not all of these parameters need to be specified. The user can choose which parameters to specify; however, if the number of specified parameters is too small, the user may receive too unspecific results and needs to repeat the approach with changes in the input. 

\begin{figure}[h!]
	\centering
	\includegraphics[width=0.6\linewidth]{Figure/Overall-concept-methods.jpg}
	\caption{Overall concept for methodical approach to systematically identify energy efficiency measures}
	\label{fig:overalconcept}
\end{figure}


\subsection{Smart Factory Building}
Firstly did research on the relation between green buildings, smart buildings, and energy efficiency. To realize how they relate each other together with energy usage, and consequently putting deeper attention on the factors that have a direct impact on consumption. We also generated an economic model based on information concerning: energy consumption and distribution in different types of buildings, prices of energy, average savings due to the implementation of smart technologies, and prices of smart technologies. 
Here we will mention some interesting energy efficiency measures that are strictly related to the energy consumption in buildings and that would be close to composing all the aspects related with that in a project like this. As figure shows a general classification of the influencing systems.

\begin{figure}[h!]
	\centering
	\includegraphics[width=0.8\linewidth]{Figure/FacturyEnergyConcumption.png}
	\caption{Smart factory building design conceptual approach}
	\label{fig:factoryenergyconsumption}
\end{figure}


\subsection{Heating, ventilation, and air conditioning (HVAC)}

\subsubsection{Heating }
Heaters are appliances whose purpose is to generate heat (i.e. warmth) for the building. This can be done via central heating. Such a system contains a boiler, furnace, or heat pump to heat water, steam, or air in a central location such as a furnace room in a home, or a mechanical room in a large building. The heat can be transferred by convection, conduction, or radiation. Since the new AHU system has heat recovery, this section takes into account that 80% of the supply air is already heated exhaust air. 

\subsubsection{Ventilation }
Ventilation is the process of changing or replacing air in any space to control temperature or remove any combination of moisture, odors, smoke, heat, dust, airborne bacteria, or carbon dioxide, and to replenish oxygen. Ventilation includes both the exchange of air with the outside as well as circulation of air within the building. It is one of the most important factors for maintaining acceptable indoor air quality in buildings. Methods for ventilating a building may be divided into mechanical/forced and natural types.The energy consumption in a single air handling unit, or process, can be noticed in the
enthalpy’s change over time.

\subsubsection{Air Conditioning }
An air conditioning system, or a standalone air conditioner, provides cooling and humidity control for all or part of a building. Air conditioned buildings often have sealed windows, because open windows would work against the system intended to maintain constant indoor air conditions. Outside, fresh air is generally drawn into the system by a vent into the indoor heat exchanger section, creating positive air pressure. The percentage of return air made up of fresh air can usually be manipulated by adjusting the opening of this vent. Typical fresh air intake is about 10%.

\subsubsection{Cooling}
cooling systems can have very high efficiencies, and are sometimes combined with seasonal thermal energy storage so that the cold of winter can be used for summer air conditioning. Common storage mediums are deep aquifers or a natural underground rock mass accessed via a cluster of small-diameter, heat-exchanger-equipped boreholes. Some systems with small storage's are hybrids, using free cooling early in the cooling season, and later employing a heat pump to chill the circulation coming from the storage. The heat pump is added-in because the storage acts as a heat sink when the system is in cooling (as opposed to charging) mode, causing the temperature to gradually increase during the cooling season.

recovery: The use of heat/enthalpy wheels and energy recovery ventilator allow them to absorb moisture from the air while at the same time cooling the air that is absorbed, to finally exhaust heated air. It is a system that allows the capacity of the HVAC system to be reduced since it can be used in both summer andwinter months. In summer it would take the heat and humidity outside the building while in winter it would exhaust the recovered heat inside the building (2015). Infrared heaters: These alternative devices can be powered electrically, by propane or with natural gas. The increase in efficiency is due to the higher impassivity the can produce compared with traditional heaters, even if in most of the cases their combustion efficiency is lower. Solar systems: Using the sun to heat water and use this thermal storage to heat the building is both cost and energy efficient. Cooling systems: When talking about this, we have to also mention the refrigeration systems. And there are three things when considering efficiency in these systems: energy usage, type and quantity of heat exhausted and refrigerant type. The air-cooled systems are the most spread used and as in many cases, the traditional equipment is not optimal in terms of energy efficiency (2017).

\begin{table}[H]
\centering
\caption{Definition of symbols \cite{Watson2018}}
\label{table:symbols}
\begin{tabular}{llll}
\hline
& 2020 & 2030 & 2050 \\ \hline
Greenhouse gas reduction & 20\% & 40\%  & 80\% \\
Energy production from renewable & 20\% & 27\% & \\
Energy efficiency improvement & 20\% & 27\% & \\ \hline
\end{tabular}

\end{table}

Energy efficiency improvement Emissions reductions in the last years have been reduced mainly thanks to progress on renewable and on energy efficiency measures. However, from the three targets of the European Union, the only one that will not be achieved by 2020 will be the one regarding improvement in energy efficiency (European Commission, 2017 \cite{European2013}). It is due to a number of barriers that go out of the scope of this project, but in general, they are related with financial uncertainty, cultural behaviors and lack of knowledge on energy efficiency measures.

We need to investigate towards several directions in order to make factories more energy efficient. General issues to be solved include:

\begin{itemize}  
\item An inter operable infrastructure that allows us to on-line monitor energy consumption down to discrete device level
\item Concrete models for energy consumption prediction at each layer e.g. device level, location, process level etc
\item Enterprise services that evaluate and assist in optimizing plans and processes dynamically based on their energy usage e.g. at production, at supply chain, etc
\item Applicability of Market driven mechanisms and close collaboration with energy providers for optimal energy usage / Integration of the distributed alternative energy resources
\item Support for software management for large scale infrastructures e.g. remote monitoring, remote diagnostics,
\end{itemize}

The application when a technician is accessing the main view of the building. Here the five different administrative domains can be accessed by pressing on the desired floor. Moreover, important events are listed on the left part of the window and the user can click on them to directly access the device emitting the alert.

\subsection{<>Conclusion }
When exceeding the level of milt machine chains, simulation techniques become predominant to master the complexity of predicting energy and resource flows for larger manufacturing systems.
Technical building services can consume considerable amounts of energy:
It is obvious that factory layout and facility optimization need careful attention in a design stage.
Just like residential buildings, production facilities need to be constructed according to state of the art building physics principles, thus minimizing energy inputs for HVAC   conditioning of the work environment, while taking into account local climate conditions.
Similar to the multi-machine strategy, production planning can be optimized at a facility wide level in order to limit the total energy consumption.




\newpage
\section{Summary}


This article includes the study on energy consumption on multiple levels; the first part of the work shows the importance of energy consumption in manufacturing industry. The research find that the big amount of energy consumed during manufacturing and the importance of energy savings. The study about the macroscopic view for the industry energy cost showed the importance of studying energy consumption and efficient improvement. 

The outline of the 
graph showed the energy consumption, literature review for energy consumption in manufacturing industry from machine to factory level. For machine level, a general method is shown for the calculation of energy consumption. An inventory of different machining processes are made and are based on the overall life cycle analysis. The possible improvement strategies are provided for a general process and for four specific processes. After that, the energy analyze for multi-machine system shows the energy connection of machines, together with the energy flows simulation. At last, review of factory level energy consumption is conducted, and investigate towards several directions in order to make factories more energy efficient for further discussion. 

Analysis of energy consumption is a major part in understanding a machine, factory. A combination of models, software and measurements is necessary to perform such an analysis. When done correctly, bottleneck machines can be discovered for manufacturing line levels or high energy consumption sub-process are brought to light in machine level. Energy improvement is done from the start of building or choosing machine/manufacturing line until the end when it is dismantled. Improvement strategies are similar over all the levels, e.g. energy management, new components or machines, etc.

An integrated effort over all the levels is necessary when we want to see a big improvement in energy consumption. Duflou et al. \cite{Duflou2012} predicts a 50\% improvement when global improvements are made in manufacturing

This study reviewed, a wide variety of considerations with relation to environmental impact reducing measures in general and energy and resource efficiency in specific has been discussed and the identified methods and techniques for analysis and system optimization.







%bibliography
\bibliographystyle{abbrv}
\bibliography{sample.bib}

\end{document}